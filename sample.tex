%----------------------------------------------------------------------------------------
%    PACKAGES AND THEMES
%----------------------------------------------------------------------------------------

\documentclass[aspectratio=169,xcolor=dvipsnames]{beamer}
\usetheme{SimplePlusPlus}


\usepackage{hyperref}
\usepackage{graphicx} % Allows including images
\usepackage{booktabs} % Allows the use of \toprule, \midrule and \bottomrule in tables
\usepackage{listings}
\usepackage{xcolor}
\usepackage{fontenc} % Allows more font options (requires Xelatex)
\usepackage{noto}
\setsansfont{Noto Sans}
\setmonofont{Noto Sans Mono}

%----------------------------------------------------------------------------------------
%    CUSTOM PYTHON THEME
%----------------------------------------------------------------------------------------
% color def
\usepackage{color}
\definecolor{darkred}{rgb}{0.6,0.0,0.0}
\definecolor{darkgreen}{rgb}{0,0.50,0}
\definecolor{lightblue}{rgb}{0.0,0.42,0.91}
\definecolor{orange}{rgb}{0.99,0.48,0.13}
\definecolor{grass}{rgb}{0.18,0.80,0.18}
\definecolor{pink}{rgb}{0.97,0.15,0.45}

% listings
\usepackage{listings}

% General Setting of listings
\lstset{
  aboveskip=1em,
  breaklines=true,
  % abovecaptionskip=-6pt,
  captionpos=b,
  escapeinside={\%*}{*)},
  frame=single,
  % numbers=left,
  % numbersep=15pt,
  % numberstyle=\tiny,
}

% 1. General Python Keywords List
\lstdefinelanguage{PythonPlus}[]{Python}{
  morekeywords=[1]{,as,assert,nonlocal,with,yield,self,True,False,None,} % Python builtin
  morekeywords=[2]{,__init__,__add__,__mul__,__div__,__sub__,__call__,__getitem__,__setitem__,__eq__,__ne__,__nonzero__,__rmul__,__radd__,__repr__,__str__,__get__,__truediv__,__pow__,__name__,__future__,__all__,}, % magic methods
  morekeywords=[3]{,object,type,isinstance,copy,deepcopy,zip,enumerate,reversed,list,set,len,dict,tuple,range,xrange,append,execfile,real,imag,reduce,str,repr,}, % common functions
  morekeywords=[4]{,Exception,NameError,IndexError,SyntaxError,TypeError,ValueError,OverflowError,ZeroDivisionError,}, % errors
  morekeywords=[5]{,ode,fsolve,sqrt,exp,sin,cos,arctan,arctan2,arccos,pi, array,norm,solve,dot,arange,isscalar,max,sum,flatten,shape,reshape,find,any,all,abs,plot,linspace,legend,quad,polyval,polyfit,hstack,concatenate,vstack,column_stack,empty,zeros,ones,rand,vander,grid,pcolor,eig,eigs,eigvals,svd,qr,tan,det,logspace,roll,min,mean,cumsum,cumprod,diff,vectorize,lstsq,cla,eye,xlabel,ylabel,squeeze,}, % numpy / math
}
% 2. New Language based on Python
\lstdefinelanguage{PyBrIM}[]{PythonPlus}{
  emph={d,E,a,Fc28,Fy,Fu,D,des,supplier,Material,PyElmt},
}


% Define colors for code elements
\definecolor{codegray}{rgb}{0.5,0.5,0.5}
\definecolor{codepurple}{HTML}{5E60CE}
\definecolor{stringblue}{HTML}{4BB4E1}
\definecolor{variableblue}{rgb}{0,0.3,0.7}
\definecolor{numbercolor}{rgb}{0.5,0,0}
\definecolor{functioncolor}{rgb}{0,0.4,0.4}
\definecolor{objectcolor}{rgb}{0.6,0.4,0}

\lstdefinestyle{mystyle}{
    commentstyle=\color{codegray},
    keywordstyle=\bfseries\color{codepurple},
    stringstyle=\color{stringblue},
    basicstyle=\ttfamily\footnotesize,
    identifierstyle=\color{variableblue},
    breakatwhitespace=false,         
    breaklines=true,                 
    captionpos=b,                    
    keepspaces=true,                 
    showspaces=false,                
    showstringspaces=false,
    showtabs=false,                  
    tabsize=2,
    frame=none
}
\lstset{
    style=mystyle,
    morekeywords=[2]{np, pd, sum},  % Objects
    morekeywords=[3]{calculate_gini, values, __init__},  % Functions
    keywordstyle=[2]{\color{codepurple}},
    keywordstyle=[3]{\color{functioncolor}}
}
% \lstdefinestyle{mystyle}{
%     % backgroundcolor=\color{backcolour},   
%     commentstyle=\color{codegray},
%     keywordstyle=\bfseries\color{codepurple},
%     % numberstyle=\tiny\color{codegray},
%     stringstyle=\color{codegreen},
%     basicstyle=\ttfamily\footnotesize,
%     breakatwhitespace=false,         
%     breaklines=true,                 
%     captionpos=b,                    
%     % keepspaces=true,                
%     % showspaces=false,                
%     showstringspaces=false,
%     showtabs=false,                  
%     tabsize=2,
%     frame =none,
% }

\lstset{style=mystyle}

% \lstset{style=colorEX}

%----------------------------------------------------------------------------------------
%    TITLE PAGE
%----------------------------------------------------------------------------------------

\title{Title}
\subtitle{Subtitle}

\author{Jakob Kauffmann}

\institute
{
    Department of Computer Science \\
    San José State University % Your institution for the title page
}
\date{\today} % Date, can be changed to a custom date

%----------------------------------------------------------------------------------------
%    PRESENTATION SLIDES
%----------------------------------------------------------------------------------------

\begin{document}
\setbeamertemplate{background}{
\includegraphics[width=\paperwidth,height=\paperheight]{backgrounds/off_white_low_bar_rainbow.pdf}
}
\begin{frame}
    % Print the title page as the first slide
    \titlepage
\end{frame}

\begin{frame}{Overview}
    % Throughout your presentation, if you choose to use \section{} and \subsection{} commands, these will automatically be printed on this slide as an overview of your presentation
    \tableofcontents
\end{frame}

%------------------------------------------------
\section{First Section}
%------------------------------------------------
\setbeamertemplate{background}{
\includegraphics[width=\paperwidth,height=\paperheight]{backgrounds/off_white_low_bar_blue.pdf}
}
\begin{frame}{Bullet Points}
    \begin{itemize}
        \item Lorem ipsum dolor sit amet, consectetur adipiscing elit
        \item Aliquam blandit faucibus nisi, sit amet dapibus enim tempus eu
        \item Nulla commodo, erat quis gravida posuere, elit lacus lobortis est, quis porttitor odio mauris at libero
        \item Nam cursus est eget velit posuere pellentesque
        \item Vestibulum faucibus velit a augue condimentum quis convallis nulla gravida
    \end{itemize}
\end{frame}

%------------------------------------------------

\begin{frame}{Blocks of Highlighted Text}
    In this slide, some important text will be \alert{highlighted} because it's important. Please, don't abuse it.

    \begin{block}{Block}
        Sample text
    \end{block}

    \begin{alertblock}{Alertblock}
        Sample text in red box
    \end{alertblock}

    \begin{examples}
        Sample text in green box. The title of the block is ``Examples".
    \end{examples}
\end{frame}

%------------------------------------------------

\begin{frame}{Multiple Columns}
    \begin{columns}[c] % The "c" option specifies centered vertical alignment while the "t" option is used for top vertical alignment

        \column{.45\textwidth} % Left column and width
        \textbf{Heading}
        \begin{enumerate}
            \item Statement
            \item Explanation
            \item Example
        \end{enumerate}

        \column{.45\textwidth} % Right column and width
        Lorem ipsum dolor sit amet, consectetur adipiscing elit. Integer lectus nisl, ultricies in feugiat rutrum, porttitor sit amet augue. Aliquam ut tortor mauris. Sed volutpat ante purus, quis accumsan dolor.

    \end{columns}
\end{frame}

%------------------------------------------------
\section{Second Section}
%------------------------------------------------

\begin{frame}{Table}
    \begin{table}
        \begin{tabular}{l l l}
            \toprule
            \textbf{Models} & \textbf{Metric 1} & \textbf{Metric 2} \\
            \midrule
            Model 1         & 0.03262           & 0.562               \\
            Model 2         & 0.15681           & 0.910               \\
            Model 3         & 0.271           & 0.296               \\
            \bottomrule
        \end{tabular}
        \caption{Table caption}
    \end{table}
\end{frame}

%------------------------------------------------

\begin{frame}{Theorem}
    \begin{theorem}[Mass--energy equivalence]
        $E = mc^2$
    \end{theorem}
\end{frame}

%------------------------------------------------

%------------------------------------------------

\begin{frame}[fragile]{Code}
\begin{block}{Some Code}
    \begin{lstlisting}[language=Python]
import numpy as np
import pandas as pd

def calculate_gini(node):
total = sum(node.values())
    if total == 0:
        return 0
    squared_probs = [(count / total) ** 2 for count in node.values()]
    gini = 1 - sum(squared_probs)
    string= "String si sjdn"
    return gini
    \end{lstlisting}
    \end{block}
\end{frame}

%------------------------------------------------

\begin{frame}{Figure}
    Uncomment the code on this slide to include your own image from the same directory as the template .TeX file.
    %\begin{figure}
    %\includegraphics[width=0.8\linewidth]{test}
    %\end{figure}
\end{frame}

%------------------------------------------------

\begin{frame}[fragile] % Need to use the fragile option when verbatim is used in the slide
    \frametitle{Citation}
    An example of the \verb|\cite| command to cite within the presentation:\\~

    This statement requires citation \cite{p1}.
\end{frame}

%------------------------------------------------

\begin{frame}{References}
    \footnotesize
    \bibliography{reference.bib}
    \bibliographystyle{apalike}
\end{frame}

%------------------------------------------------
\setbeamertemplate{background}{
\includegraphics[width=\paperwidth,height=\paperheight]{backgrounds/off_white_low_bar_blue.pdf}
}
\begin{frame}
    \Huge{\centerline{\textbf{The End}}}
\end{frame}

%----------------------------------------------------------------------------------------

\end{document}